\section{Overview}
% Researchers universally assume that participants use the input they receive (i.e., the value of each option) to make a choice.
Decades of decision-making research have shown that context can systematically affect choice. In decision-making experiments, researchers present participants with a finite set of options on each trial and ask them to select a single option based on either an internal (e.g., most preferable) or external (e.g., largest shape) criterion. Decision-making research spans multiple fields, including psychology, neuroscience, economics, marketing, and political science. In economics, for example, researchers have developed models based on the idea that, while preferences may vary from moment to moment, people generally make rational choices in any given choice setting. In psychology and marketing, however, researchers have identified a set of phenomena that violate such assumptions, by showing that choices can vary with the \textit{choice set}, or the menu of available options. This class of phenomena is known as \textit{context effects}.

Context effects are interesting to decision-making researchers because they violate properties exhibited by large classes of choice models, such as Independence of Irrelevant Alternatives (IIA) \parencite{ray1973independence} and regularity \parencite{mackay1995probabilistic,marley1989random}. IIA states that the likelihood of selecting one option over another is invariant of other options available. Regularity states that the probability an option is chosen cannot increase upon the addition of new options to a choice set. IIA and regularity are also properties of Luce's Choice Axiom, a highly influential model of stochastic choice \parencite{luceChoiceAxiomTwenty1977a, luce1959individual}. 

One notable context effect, the attraction effect, occurs when the choice share of a \textit{target} option increases upon the inclusion of a similar but inferior \textit{decoy} option \parencite{huberAddingAsymmetricallyDominated1982d}. Another finding, the repulsion effect, occurs when a decoy boosts the choice share of a dissimilar \textit{competitor} option rather than the target \parencite{simonson2014vices}. The repulsion effect is an empirical reversal of the attraction effect. 

Context effects, originally studied in preferential choice, have been recently demonstrated simple perceptual choice \parencite{trueblood2013not,spektorWhenGoodLooks2018b,liaoInfluenceDistanceDecoy2021,spektorRepulsionEffectPreferential2022,yearsleyContextEffectsSimilarity2022,truebloodPhantomDecoyEffect2017c, turnerCompetingTheoriesMultialternative2018a, evansImpactPresentationOrder2021}. This idea is theoretically interesting because it suggests that context effects are a theoretical primitive rather than simply a feature of high-level consumer choice \parencite{trueblood2013not}. 

As suggested by the title, this dissertation explores various forms of context dependence in both perceptual and preferential choice. In particular, I explore the attraction and repulsion effects. The goal of this dissertation is to further understand why these effects occur in perceptual choice, by employing well-studied statistical models from the psychology literature. Additionally, this dissertation sets out to differentiate perceptual processes from decision-making processes in context effects (specifically, the attraction and repulsion effects).

As I will discuss throughout this dissertation but particularly in Chapter 2, recent work has demonstrated inconsistency in context effects, particularly in perceptual choice. I use behavioral experiments and statistical modeling in an attempt to reconcile these inconsistencies.

This dissertation is structured as follows. In Chapter 2, I develop and test a statistical model of perceptual variability when applied to context effects. In Experiment 1, I first show that the types of stimuli used in perceptual choice context effects experiments are easily confusable and vary systematically with theoretically relevant properties of the stimuli. In Experiment 2, I use the results of a high-powered psychophysics experiment to show that the repulsion effect, but not the attraction effect, is naturally predicted by this statistical model. In Chapter 3, I further test the statistical model by applying it to best-worst choice. In Chapter 4, I generalize the paradigm and model to preferential choice. Finally, in Chapter 5, I use a perceptual choice experiment to show that stimulus comparability affects choice, even when the decoy is equally similar to both focal options. In Chapter 6, I summarize the findings of the dissertation, their implications, and discuss future directions for research in this domain. 

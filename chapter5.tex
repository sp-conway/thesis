\section{Introduction}

Thus far in the dissertation, I have used stimuli from the attraction and repulsion effect to explore perceptual and decisional processes in both perceptual and preferential choice. I showed that stimuli which are more similar to one another, and are thus more easily comparable, generate valuations with stronger correlations. This result holds across both perceptual choice (Chapter 2) and preferential choice (Chapter 4). I define the comparison process as a cognitive operation where a participant determines the relative difference between two options on a choice set, typically (though not necessarily) on a single dimension. 

These correlations can produce the repulsion effect \parencite{spektorWhenGoodLooks2018b,simonson2014vices} because the decoy option, whose value is tightly correlated with the target, occasionally exceeds the target in perceived value and thus "steals" choice shares. 

In this chapter, I directly manipulate stimulus comparability in a perceptual choice task in an attempt to understand the relationship between comparability and choice. I first review previous literature on comparability and then show the results of a simple perceptual choice experiment.

\subsection{Previous Literature on Comparability}

Other researchers have studied the comparison process in decision-making, particularly in high-level choice (e.g., preferential). I first discuss the preferential choice work before transitioning to previous research on perceptual choice. \textcite{changWhichCompromiseOption2008} tested the compromise effect (see Chapter 1) by varying the display. In the compromise effect, a “middle” ground option decreases the choice share of two dissimilar, “extreme” options. \textcite{changWhichCompromiseOption2008} displayed the options either by-alternative format, where option names are listed as columns while attribute values are listed as rows, or by-attribute, where option attributes are columns while option names are rows. The former display makes it more difficult to compare options on a single attribute, while the latter makes it easier. \textcite{changWhichCompromiseOption2008} found that listing options by-attribute increased the choice share of the compromise option, relative to a by-alternative display. 

\textcite{cataldoComparisonProcessAccount2019b} replicated this result, also finding that a by-alternative format nullified the attraction effect. The authors attributed this result to a “flexible comparison process”, where the comparison strategy is to some extent dictated by display format. According to this account, the by-attribute format increases the ease of target-decoy comparisons relative to the by-alternative format \footnote{\textcite{hasan2025registered} failed to replicate these results, albeit with slightly different decoy types.}. \textcite{cataldoReversingSimilarityEffect2018b} show that presenting options in a format that encourages within-dimension comparisons on pairs of options can reverse the well-studied similarity effect.

\textcite{noguchi2014attraction} studied context effects using eye-tracking, showing that people tend to compare pairs of options on a single attribute, and that this appears to drive the attraction, similarity, and compromise effect. In their study, participants' eye movements showed that they were more likely to transition between options on a single dimension than they were to transition between dimensions within a single option. They also found that transitions between two options are negatively related to the choice share of a third option.

\textcite{hayes2024attribute} manipulated attribute comparability, such that the dimensions of each option were either measured in the same unit (high comparability, e.g., 0-10 ratings) or in different units (low comparability, e.g., CPU speed vs. RAM for laptops). They found that the attraction effect only occurred in the low comparability condition. 

Hsee and colleagues \parencite{hseeEvaluabilityHypothesisExplanation1996,hseeLessBetterWhen1998,hseeWillProductsLook1998,hsee1999preference} have also shown that the comparison of options affects consumer behavior. For example, they repeatedly showed that participants’ evaluation of a given option can change with the addition of a reference point (i.e., lower valued options improve with a high reference point and vice versa). That is, participants’ judgments are different when options are evaluated jointly, compared to separately \parencite{hsee1999preference}. 

Many theoretical accounts of decision-making rely on the comparison process as well. According to \textcite{trueblood2014multiattribute}'s Multiattribute Linear Ballistic Accumulator Model (MLBA), each option accumulates evidence through pairwise comparisons to all other available options. This comparison is modulated by several processes, such as distance in attribute space and extremeness aversion. \textcite{roeMultialternativeDecisionField2001a}'s Multialternative Decision Field Theory (MDFT) model also assumes that options accumulate evidence through comparison, and that options nearby in attribute space are in greater competition than those further away. Other evidence accumulation models incorporate similar mechanisms \parencite{usherLossAversionInhibition2004a,noguchiMultialternativeDecisionSampling2018a,wollschlager2NaryChoiceTree2012a,landry2021pairwise} (c.f. \textcite{bhatiaAssociationsAccumulationPreference2013b,bergnerVAMPVotingAgent2019b}). 

\textcite{trueblood2022attentional} argued that options that are more similar garner more attention in the comparison process, and showed, via a simple modeling analysis, that "similarity-guided attention" can explain well-known context effects.

\subsection{Comparability Effects in Perceptual Choice}
There has been other research, albeit relatively limited, on the comparison process in perceptual decision-making. Much of this work has focused on the spatial layout of the options and its effect on context effects.

\textcite{trueblood2022attentional} re-analyzed previous perceptual choice context effect data \parencite{trueblood2015fragile} by examining the order of the options on the screen. They found that the attraction effect was strongest when the target and decoy were next to each other, while the effect was weak (or even nullified) when the options were separated spatially. Their conclusion, supported by a modeling analysis, was that people tend to compare pairs of options which are spatially closer to one another more often than pairs further away from one another. This result may seem obvious, but previous researchers have largely ignored the order of options in choice, generally collapsing over order in all analyses.

\textcite{evansImpactPresentationOrder2021} found a similar result in perceptual choice, though in their experiment the options were separated both spatially and temporally. In their experiment, participants saw three rectangles, presented sequentially, and selected the largest rectangle after all stimuli were presented. They found that orders in which the target and decoy were presented in the latter two positions elicited an attraction effect, whereas orders in which the competitor and decoy were presented in the latter two positions tended to elicit a repulsion effect. They interpreted their results as evidence that the comparison process can be altered through spatial and temporal properties of the stimuli.

Another interpretation of their results, and those of \textcite{trueblood2022attentional}, is that by altering the location and timing of the stimuli, the researchers are also altering the comparability. In the perceptual model of Chapter 2, increased comparability is represented by an increase in perceptual correlation. As shown previously, this perceptual correlation can create a repulsion effect by allowing the decoy to more easily "steal" choice shares from the target.

The experiments of \textcite{evansImpactPresentationOrder2021} and \textcite{trueblood2015fragile} are interesting and have to much to tell us about the role of comparability in decision-making. However, I wish to isolate the effect of comparability from context effects. To do so, I conducted an experiment where participants saw three rectangles at a time and selected the largest one. On critical trials, two of these rectangles were equally large but oriented differently (i.e., \textit{focal} rectangles, as in Experiments 1, 2, and 3). A third \textit{decoy} option was a square, and thus equally similar to either option. However, I altered the stimulus presentation format such that on some trials the decoy was nearer the wide (tall) rectangle and thus easier to compare to the wide (tall) rectangle. Based on the results of the modeling from Chapter 2, this should increase the correlation between the comparable options and result in a \textit{decrease} in choice share relative to the less comparable focal option. These results are generally borne out in the data, albeit with limitations which will be adressed in future word.

\section{Experiment 5}
Experiment 5 adresses the effects of comparability in perceptual choice. To do so, I incorporated \textit{symmetrically dominated decoys}.

\subsection{Methods}

\subsubsection{Participants}
$231$ undergraduate students at the University of Massachusetts Amherst participated in exchange for course credit. $17$ participants' data were removed from all analyses becauses they failed to achieve at least $80\%$ correct on catch trials (see below), leaving final sample size of $N=214$. 









A large body of research in psychology has shown that context can systematically affect choice. These \textit{context effects} are in violation of classical choice models and are both practically and theoretically interesting. Recent work has demonstrated context effects in simple perceptual choice,  findings that suggest that these results are not restricted to high-level (e.g., consumer) choice. This dissertation explores the attraction and repulsion effects, two related context effects, with a particular emphasis on perceptual choice. Chapter 1 introduces the attraction and repulsion effects and reviews the empirical and theoretical literature surrounding them. Chapter 2 tests the ability of perceptual and decisional processes to account for these effects via a Thurstonian choice model. Chapter 3 uses this model to make predictions for best-worst choice and tests them empirically. Chapter 4 generalizes the Thurstonian model and experimental paradigm from Chapter 2 to consumer choice. Chapter 5 tests the idea that an effect similar to that of Chapter 2 can be generated through ease of inter-option comparability in perceptual choice. Chapter 6 summarizes and discusses these results. 
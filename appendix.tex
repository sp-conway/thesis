\chapter{Bayesian Logistic Regression Model of 2AFC Discriminability from Experiment 1}
I analyzed the 2AFC data from Experiment 1 using Bayesian Hierarchical Logistic Regression. 

Recall that in this experiment, participants were presented with three stimuli (target, competitor, and decoy rectangles). They were then asked to select the largest rectangle out of a pair of two of these options. In other words, there are three trial types: target/competitor (TC), target/decoy (TD), and competitor/decoy (CD). As discussed in the main text, I remove the TC trials from all substantive analyses, as well as the $TDD=0\%$ trials. 

\section{Model Details} 

The model predicts the probability of discriminating the target/competitor from the decoy option. According to the model, discrimination $D$ for participant $i$ on trial $j$ is:
\begin{align}
    D_{ij} \sim Bernoulli(\theta_{ij})
\end{align}

To compute $\theta_{ij}$, I first compute $\eta_{ij}$ from linear combination of the relevant variables.

\begin{align}
    \eta_{ij} &= (\beta_{0} + S_{0_{i}}) + (\beta_{\mathrm{or}} + S_{\mathrm{or}_{i}}) \cdot \mathrm{or}_{j} + (\beta_{\mathrm{horiz}} + S_{\mathrm{horiz}_{i}}) \cdot \mathrm{horiz}_{j} \\
    &\quad + (\beta_{\mathrm{TD}} + S_{\mathrm{TD}_{i}}) \cdot \mathrm{TD}_{j} + (\beta_{\mathrm{TDD5}} + S_{\mathrm{TDD5}_{i}}) \cdot \mathrm{TDD5}_{j} + (\beta_{\mathrm{TDD9}} + S_{\mathrm{TDD9}_{i}}) \cdot \mathrm{TDD9}_{j} \\
    &\quad + (\beta_{\mathrm{TDD14}} + S_{\mathrm{TDD14}_{i}}) \cdot \mathrm{TDD14}_{j} + (\beta_{\mathrm{TDD5xTD}} + S_{\mathrm{TDD5xTD}_{i}}) \cdot \mathrm{TDD5}_{j} \cdot \mathrm{TD}_{j} \\
    &\quad + (\beta_{\mathrm{TDD9xTD}} + S_{\mathrm{TDD9xTD}_{i}}) \cdot \mathrm{TDD9}_{j} \cdot \mathrm{TD}_{j} + (\beta_{\mathrm{TDD14xTD}} + S_{\mathrm{TDD14xTD}_{i}}) \cdot \mathrm{TDD14}_{j} \cdot \mathrm{TD}_{j}
\end{align}

All $\beta$ terms are fixed effects, and all $S$ terms are random (participant) effects. $\beta_{or}$ is the fixed effect of orientation, where $\text{or}_{j}$ is a dummy variable which $=0$ if the the target and decoy are taller than wide and $=1$ if the target and decoy are wider than tall. $\beta_{TD}$ is the fixed effect of comparison, where $TD_{j}$=0 for CD trials and $TD_{j}=1$ for TD trials.  The $TDD$ variable has 4 levels ($2\%$, $5\%$, $9\%$, and $14\%$), so I include three dummy variables ($TDD5$, $TDD9$, and $TDD14$) and treat $2\%$ as the reference level for $TDD$. I also include the interaction to capture the additional boost / decrement to $TD$ over $TC$ trials at each level of $TDD$. 

$\eta_{ij}$ is then transformed to the probability scale using the logit function:

\begin{align}
    \theta_{ij}=\frac{1}{1+e^{-\eta_{ij}}}
\end{align}

\section{Prior Distributions on Parameters}

\begin{itemize}
    \item $\beta_{0} \sim \mathcal{N}(0,5)$
    \item $\beta_{or} \sim \mathcal{N}(0,5)$
    \item $\beta_{horiz} \sim \mathcal{N}(0,5)$
    \item $\beta_{TD} \sim \mathcal{N}(0,5)$
    \item $\beta_{TDD5} \sim \mathcal{N}(0,5)$
    \item $\beta_{TDD9} \sim \mathcal{N}(0,5)$
    \item $\beta_{TDD14} \sim \mathcal{N}(0,5)$
    \item $\beta_{TDD5xTD} \sim \mathcal{N}(0,2.5)$
    \item $\beta_{TDD9xTD} \sim \mathcal{N}(0,2.5)$
    \item $\beta_{TDD14xTD} \sim \mathcal{N}(0,2.5)$
    \item $S_{0_{i}} \sim \mathcal{N}(0, \sigma_{S_0})$
    \item $S_{or_{i}} \sim \mathcal{N}(0, \sigma_{S})$
    \item $S_{horiz_{i}} \sim \mathcal{N}(0, \sigma_{S})$
    \item $S_{TD_{i}} \sim \mathcal{N}(0, \sigma_{S})$
    \item $S_{TDD5_{i}} \sim \mathcal{N}(0, \sigma_{S})$
    \item $S_{TDD9_{i}} \sim \mathcal{N}(0, \sigma_{S})$
    \item $S_{TDD14_{i}} \sim \mathcal{N}(0, \sigma_{S})$
    \item $S_{TDD5xTD_{i}} \sim \mathcal{N}(0, \sigma_{S})$
    \item $S_{TDD9xTD_{i}} \sim \mathcal{N}(0, \sigma_{S})$
    \item $S_{TDD14xTD_{i}} \sim \mathcal{N}(0, \sigma_{S})$
    \item $\sigma_{S_{0}} \sim LogNormal(0,2.5)$
    \item $\sigma_{S} \sim LogNormal(0,2.5)$
\end{itemize}

Note that the model assumes equal variance for all random effect distributions aside from the random intercepts.

\section{Modeling Results}
The model was coded in Stan \parencite{carpenter2017stan} and implemented using the RStan package \parencite{rstan}. The sampler ran 5 chains, each for 2500 iterations. Posterior diagnostics indicated that the sampler converged.

\subsection{Parameter estimates.}

Table XXX shows parameter estimates, including means and $95\%$ credible intervals. 
\begin{table}[ht]
    \centering
    \begin{tabular}{lrrr}
        \toprule
        Parameter & M & SD & CI low & CI high \\
        \midrule
        $\beta_{0}$ & 0.44 & 0.06 & 0.32 & 0.57 \\
        $\beta_{or}$ & -0.54 & 0.06 & -0.66 & -0.43\\
        $\beta_{horiz}$ & 0.23 & 0.06 & 0.12 & 0.34\\
        $\beta_{TD}$ & 0.17 & 0.08 & 0.01 & 0.33\\
        $\beta_{TDD5}$ & 0.40 & 0.08 & 0.23 & 0.56\\
        $\beta_{TDD9}$ & 0.81 & 0.09 & 0.64 & 0.98\\
        $\beta_{TDD14}$ & 1.45 & 0.10 & 1.25 & 1.64\\
        $\beta_{TDD5xTD}$ & 0.14 & 0.12 & -0.09 & 0.36\\
        $\beta_{TDD9xTD}$ & 0.61 & 0.13 & 0.36 & 0.85\\
        $\beta_{TDD14xTD}$ & 0.79 & 0.15 & 0.50 & 1.10\\
        $\sigma_{S_{0}}$ & 0.20 & 0.04 & 0.12 & 0.29 \\
        $\sigma_{S}$ & 0.34 & 0.03 & 0.29 & 0.39 \\
    \bottomrule 
    \end{tabular}
    \caption{Table XXX. Parameter estimates for Bayesian Hierarchical Logistic Regression from Experiment 1 Data, including means, standard deviations, and $95\%$ Credible Intervals.}
    \label{tab:e1_params}
    \end{table}
 \end{table}
    
 Inference by examining the posterior distributions of the fixed effect parameters (i.e., all $\beta$ values). 

\chapter{Experiment 2: Bayesian Hierarchical Modeling of Circle Estimation Data}

I analyzed the circle estimation data (Experiment 2) using the multivariate perceptual model first presented in Chapter 2. 

\section{Model Details}

The model predicts that, for participant $i$ on trial $j$, the vector of perceived areas $\mathbf{X}_{ij}$ is sampled from a multivariate normal distribution with parameters $\boldsymbol{\mu}_{ij}$ and $\boldsymbol{\Sigma}$. That is,
\begin{align}
    \mathbf{X}_{ij} \sim \mathcal{N}(\boldsymbol{\mu}_{ij},\boldsymbol{\Sigma})
\end{align}

Using Bayesian statistical modeling, I simultaneously estimated the parameters  $\mathbf{\mu$ and $\boldsymbol{\Sigma}$ for the model outlined in Chapter 2. Note that I allow $\boldsymbol{\mu}$ to vary sytematically over trials and participants, but $\Sigma$ remains constant. I estimate $\boldsymbol{\mu}$ using hierarchical regression while allowing the components of $\boldsymbol{\Sigma}$ (i.e., $\mathbf{\sigma_{T}}$, $\mathbf{\sigma_{C}}$, $\mathbf{\srigma_{D}}$, $\rho_{TD}$, $\rho_{TC}$, $\rho_{CD}$) to vary freely. 

I estimated the model separately for the triangle and horizontal condition. I first walk through the computation of $\mathbf{\mu$}, followed by the computation $\boldsymbol{\Sigma}$. I also show the prior distributions on each parameter separately for each of these components, and then I explain the modeling procedure and results. Note that the model predicts mean-centered log-transformed estimated area.

\subsection{\texorpdfstring{$\boldsymbol{\mu}$} Parameterization}
I use the model to predict the mean area $\mu_{ijk}$ for the $i$th participant on the $j$th trial for the $k$th stimulus. There are $k=3$ stimuli on each trial. If $k=1$, the stimulus is the competitor; If $k=2$, the stimulus is the competitor; If $k=3$, the stimulus is the decoy. Thus, $\boldsymbol{\mu}$ can be broken down into $\mu_{T}_{ij}$, $\mu_{C}_{ij}$, $\mu_{D}_{ij}$,. Note that some parameters are common to all stimuli (e.g., the effects of diagonal or orientation), while others are common only to a particular option (e.g., the effect of competitor vs. decoy vs. competitor). 

$\mu_{T}_{ij}$ is computed as:

\begin{align}
    \mu_{T}_{ij}=(S_{0}_{i} + \beta_{0}) + \beta_{or}*\mathrm{or}_{T}_{ij} + \beta_{\mathrm{diag}2}*\mathrm{diag}2_{ij}+ \beta_{\mathrm{diag}3}*\mathrm{diag}3_{ij} + \beta_{\mathrm{tdd}5}*\mathrm{tdd}5_{ij} + \beta_{\mathrm{tdd}9}*\mathrm{tdd}9_{ij} + \beta_{\mathrm{tdd}14}*\mathrm{tdd}14_{ij}
    \label{circle_mu_eqn1}
\end{align}

$S_{0}_{i}$ is a random intercept for participant $i$. $\beta_{0}$ is the fixed intercept. $\beta_{or}$ is the fixed effect of orientation, where $\or_{T}_{ij}$ is a dummy variable which $=0$ if the the competitor is taller than wide and $=1$ if the competitor is wider than tall. $\beta_{diag2}$ is the fixed effect of the middle diagonal, which $=1$ if the all stimuli on the trial fall on the middle diagonal and $=0$ otherwise. $\beta_{diag2}$ is the fixed effect of the upper diagonal, which $=1$ if all stimuli on the trial fall on the upper diagonal and $=0$ otherwise. $\beta_{tdd5}$ is the fixed effect of TDD 5, and $tdd5_{ij}$ is a dummy variable which $=1$ if $TDD=5\%$ and $=0$ otherwise. $\beta_{tdd9}$ is the fixed effect of TDD 9, and $tdd9_{ij}$ is a dummy variable which $=1$ if $TDD=9\%$ and $=0$ otherwise. $\beta_{tdd14}$ is the fixed effect of TDD 14, and $tdd14_{ij}$ is a dummy variable which $=1$ if $TDD=14\%$ and $=0$ otherwise. 

$\mu_{C}_{ij}$ is computed as:

\begin{align}
    \mu_{C}_{ij}=(S_{0}_{i} + \beta_{0}) + \beta_{\mathrm{or}}*\mathrm{or}_{C}_{ij} + \beta_{\mathrm{diag}2}*\mathrm{diag}2_{ij}+ \beta_{\mathrm{diag}3}*\mathrm{diag}3_{ij} + \beta_{\mathrm{tdd}5}*\mathrm{tdd}5_{ij} + \beta_{\mathrm{tdd}9}*\mathrm{tdd}9_{ij} + \beta_{\mathrm{tdd}14}*\mathrm{tdd}14_{ij} + \beta_{\mathrm{comp}}
    \label{circle_mu_eqn2}
\end{align}

$S_{0}_{i}$ is a random intercept for participant $i$. $\beta_{0}$ is the fixed intercept. $\beta_{or}$ is the fixed effect of orientation, where $\or_{C}_{ij}$ is a dummy variable which $=0$ if the the competitor is taller than wide and $=1$ if the competitor is wider than tall. $\beta_{diag2}$ is the fixed effect of the middle diagonal, which $=1$ if the all stimuli on the trial fall on the middle diagonal and $=0$ otherwise. $\beta_{diag2}$ is the fixed effect of the upper diagonal, which $=1$ if all stimuli on the trial fall on the upper diagonal and $=0$ otherwise. $\beta_{tdd5}$ is the fixed effect of TDD 5, and $tdd5_{ij}$ is a dummy variable which $=1$ if $TDD=5\%$ and $=0$ otherwise. $\beta_{tdd9}$ is the fixed effect of TDD 9, and $tdd9_{ij}$ is a dummy variable which $=1$ if $TDD=9\%$ and $=0$ otherwise. $\beta_{tdd14}$ is the fixed effect of TDD 14, and $tdd14_{ij}$ is a dummy variable which $=1$ if $TDD=14\%$ and $=0$ otherwise. $\beta_{comp}$ is a paramter that reflects the possibility of estimation bias for the competitor.

$\mu_{D}_{ij}$ is computed as:

\begin{align}
    \mu_{D}_{ij}=(S_{0}_{i} + \beta_{0}) + \beta_{\mathrm{or}}*\mathrm{or}_{D}_{ij} + \beta_{\mathrm{diag}2}*\mathrm{diag}2_{ij}+ \beta_{\mathrm{diag}3}*\mathrm{diag}3_{ij} + (\beta_{\mathrm{tdd}5} + \beta_{\mathrm{tdd}5D})*\mathrm{tdd}5D_{ij} + (\beta_{\mathrm{tdd}9} + \beta_{\mathrm{tdd}9D})*\mathrm{tdd}9D_{ij} + (\beta_{\mathrm{tdd}14} + \beta_{\mathrm{tdd}14D})*\mathrm{tdd}14D_{ij}
    \label{circle_mu_eqn3}
\end{align}

$S_{0}_{i}$ is a random intercept for participant $i$. $\beta_{0}$ is the fixed intercept. $\beta_{or}$ is the fixed effect of orientation, where $\or_{D}_{ij}$ is a dummy variable which $=0$ if the the decoy is taller than wide and $=1$ if the decoy is wider than tall. $\beta_{diag2}$ is the fixed effect of the middle diagonal, which $=1$ if the all stimuli on the trial fall on the middle diagonal and $=0$ otherwise. $\beta_{diag2}$ is the fixed effect of the upper diagonal, which $=1$ if all stimuli on the trial fall on the upper diagonal and $=0$ otherwise. $\beta_{tdd5D}$ is the fixed effect of TDD=5, and $tdd5D_{ij}$ is a dummy variable which $=1$ if $TDD=5\%$ for the decoy and $=0$ otherwise. $\beta_{tdd9D}$ is the fixed effect of TDD 9 for the decoy, and $TDD9D_{ij}$ is a dummy variable which $=1$ if $TDD=9\%$ and $=0$ otherwise. $\beta_{TDD14D}$ is the fixed effect of TDD 14 for the decoy, and $TDD14D_{ij}$ is a dummy variable which $=1$ if $TDD=14\%$ and $=0$ otherwise. 

Note that there is a common set of parameters for each level of TDD and additional set of parameters for each level of TDD that only apply to the decoy. In the data, it was clear that participants often adjusted the target and competitor relative to the decoy. In other words, even though the physical size of both target and competitor remains constant across TDD, participants' \textit{estimation} of their size varied with TDD. The inclusion of a separate set of parameters for TDD that only apply to the decoy allows for a "deflection" of the decoy size, relative to target and competitor size. 

Note the following reference point for each variable:
\begin{itemize}
    \item TDD: $2\%$
    \item Orientation: taller than wide
    \item Diagonal: lower
    \item Stimulus: target
\end{itemize}

In other words, the $\beta_{0}$ and $S_{0}_{i}$ parameters capture the fixed and random effects, respectively, for a tall target on the lower diagonal at $2\%$ TDD, and all other parameters reflect deflections from this.

\subsubsection{Prior Distributions on Parameters}
Below are shown the following prior distributions on each parameter relevant to $\boldsymbol{\mu}$:
\begin{itemize}
    \item $\beta_{0} \sim \mathcal{N}(0,5)$
    \item $\beta_{or} \sim \mathcal{N}(0,5)$
    \item $\beta_{diag2} \sim \mathcal{N}(0,5)$
    \item $\beta_{diag3} \sim \mathcal{N}(0,5)$
    \item $\beta_{TDD5} \sim \mathcal{N}(0,5)$
    \item $\beta_{TDD9} \sim \mathcal{N}(0,5)$
    \item $\beta_{TDD14} \sim \mathcal{N}(0,5)$
    \item $\beta_{TDD5D} \sim \mathcal{N}(0,5)$
    \item $\beta_{TDD9D} \sim \mathcal{N}(0,5)$
    \item $\beta_{TDD14D} \sim \mathcal{N}(0,5)$
    \item $\beta_{comp} \sim \mathcal{N}(0,5)$
    \item $S_{0}_{i} \sim \mathcal{N}(0,\sigma_{S}_{0})$
    \item $\sigma_{S}_{0} \sim Half-Cauchy(0,2.5)$
\end{itemize}

\section{$\boldsymbol{\Sigma}$ Parameterization}
As discussed in Chapter 2, $\Sigma$ is a positive semi-definite 3x3 covariance matrix computed as:

\begin{align}
   \boldsymbol{\Sigma}=S\boldsymbol{\Omega}S
   \label{eqn:Sigma}
\end{align}

where $S$ is a diagonal matrix consisting of: 

\begin{align}
   \begin{pmatrix}
      \sigma_{T} & 0 & 0 \\
      0 & \sigma_{C} & 0 \\
      0 & 0 & \sigma_{D} \\
   \end{pmatrix}
   \label{eqn:S}
\end{align}

and $\boldsymbol{\Omega}$ is a correlation matrix:

\begin{align}
   \begin{pmatrix}
      1 & \rho_{TC} & \rho_{TD} \\
      \rho_{TC} & 1 & \rho_{CD} \\
      \rho_{TD} & \rho_{CD} & 1 \\
   \end{pmatrix}
   \label{eqn:O}
\end{align}

Estimation of $S$ is straightforward. I simply freely estimate the three standard deviation parameters $\sigma_{T}$, $\sigma_{C}$, and $\sigma_{D}$.

To estimate $\boldsymbol{\Omega}$, I used the LKJ distribution \parencite{lewandowski2009generating} to set priors on the Cholesky factorization of the correlation matrix $\Omega$. This was done to ensure that the resulting variance-covariance matrix $\boldsymbol{\Sigma}$ is positive semi-definite, a requirement of the multivariate Gaussian distribution. The critical inferences, however, are performed on the off-diagonal elements $\rho_{TC}$, $\rho_{TD}$, $\rho_{CD}$ in each display condition. I set priors on the $\sigma$ parameters using the Half-Cauchy distribution \parencite{gelman2006prior}. 

\subsubsection{Prior Distributions on Parameters}
Below are shown the following prior distributions on each parameter relevant to $\boldsymbol{\Sigma}$.
\begin{itemize}
    \item $\sigma_{T} \sim Half-Cauchy(0,2.5)$
    \item $\sigma_{C} \sim Half-Cauchy(0,2.5)$
    \item $\sigma_{D} \sim Half-Cauchy(0,2.5)$
    \item $\boldsymbol{\Omega} \sim LKJCorr(1)$
\end{itemize}

\section{Modeling Results}
The model was implemented as a Bayesian model using the Stan programming language \parencite{carpenter2017stan} using the cmdstanr interface \parencite{cmdstanr}, which allows the R programming language to interface with the cmdstan implementation of Stan.

I ran the model for 2500 iterations (not including warm-up) with 4 chains for each display condition. Posterior diagnostics indicated that the sampler converged.

Below I show parameter estimates for each display condition and relevant parameter. I exclude the estimates of the participant effects $S_{0}$ for brevity.



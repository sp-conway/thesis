\section{General Discussion}

Decisions, even simple ones, may be impacted by context. Context effects occur when the relative choice for a particular option varies with situational properties. In particular, the set of other available affects choice. The attraction effect occurs when a decoy option "boosts" the choice share of a similar, superior target option. The repulsion effect, a reversal of the attraction effect, occurs when the inclusion of the decoy causes people to select the dissimilar competitor more than the target. 

This dissertation has explored context dependence in choice, largely by studying the attraction and repulsion effect (Chapters 2-4) but also through studying context dependence induced by the presentation format of options in simple perceptual choice (Chapter 5).

In Chapter 2, I used a Thurstonian perceptual choice model to measure the correlation between valuations in the attraction and repulsion effect. In Experiment 2, I collected valuations (area estimations for simple rectanges) and choices (where participants selected the largest rectangle from each ternary choice set). I connected valuations to choices using the Thurstonian model. I showed that, through the $\rho_{TD}$ parameter, this model can parsimoniously account for the repulsion effect without invoking higher-level decision processes (Experiment 2). The model cannot, however, account for the attraction effect, suggesting that higher-level decision processes may be required to explain this effect. I also demonstrated systematic discriminability issues in a 2AFC task, using the same experimental stimuli, in accordance with the Thurstonian model (Experiment 1).

In Chapter 3, Experiment 3, I showed that the model makes an interesting prediction for best-worst choice experiments. In best-worst choice, participants select their most and least preferred options from a given choice set. Given the stimuli from Experiment 2 (a set of target, competitor, and decoy rectangles in a perceptual choice experiment), the model predicts a non-monotonic relationship between best and worst choice probabilities. Specifically, the model predicts that the target is both less likely to be chosen as worst, compared to the competitor, and also less likely to be chosen as best. This effect is not predicted by the maxdiff model, a highly influential of best-worst choice. Thus, I have identified another form of context dependence: whether participants are selecting the best option or the worst option from a particular choice set.

In Chapter 4, Experiment 4, I generalized the paradigm and model from Chapter 2 to preferential choice. On each experimental trial, participants saw three consumer products (e.g., microwave ovens, laptops) and assigned each one a viable selling price. In later experimental trials, they saw the same three options and selected the option they most preferred. I showed that the similarity, and comparability, of the target and decoy options, appears to generalize across choice types and can be reliably measured using Pearson correlations.  I also replicated previous researchers' choice results \parencite{banerjeeFactorsThatPromote2024}, albeit with limitations. For example, the experiment did not systematically decoy position nor did it include binary choices to combat this limitation. 

The model from Experiment 4 is able to qualitatively account for the repulsion effect, there is one crucial limitation here; the model accounts for the effect through the correlation between target and decoy evaluations, which causes the decoy to take choice shares away from the target. In preferential choice, and unlike perceptual choice, participants seldom if ever select the decoy. In the absence of these correlations, or if all correlations are equal (i.e., $\rho_{TD}=\rho_{TC}=\rho_{CD}$), the model will be unable to predict the effect\footnote{See the diagonal line from Figure~\ref{fig:3d_model} for evidence of this result.}. Thus, this form of the repulsion effect may be due to higher-level decision processes. Nonetheless, the demonstration of target-decoy correlations, and indeed target-competitor correlations, is a novel and interesting result. 

Though other researchers have proposed correlations as a measure of the similarity between options in a choice set \parencite{kamakura1984predicting,natenzon2019random}, these studies was the first (to my knowledge) to systematically measure these correlations using valuations, incorporate them into a Thurstonian choice model, and connect this model to choices obtained from the same experimental participants.

In Chapter 5, Experiment 5, I demonstrated a new form of context dependence, where choice systematically varies based on option comparability. Given a \textit{symmetrically dominated decoy} option, placing a focal target option in a nearby position, such that participants can more easily compare it to the decoy, \textit{decreases} the target's choice share. This is similar to the standard repulsion effect, with the important caveat that both focal options are equally perceptually similar to the decoy. 

There were limitations to this experiment. In particular, given the critical "two-aligned" condition, the target and decoy were only ever in the first and second positions. Future work should fix this experimental design flaw, while also including different configurations.

The results of this dissertation also have important methodological considerations. In particular, I argue that researchers should carefully consider the assumptions made when designing and analyzing experiments. For example, the experiments of \textcite{spektorWhenGoodLooks2018b} contain a crucial assumption: that stimulus discriminability issues do not systematically affect target or competitor choice, and that the repulsion effect observed in these experiments is a qualitative reversal of the attraction effect rather than just an empirical one. I also showed, in Chapter 3, that the independence assumption of the maxdiff choice model is incorrect (at least in some cases) and a failure to consider whether the stimuli of a given experiment can cause these violations may lead to incorrect conclusions about participants' preferences. The results of Chapter 5 show that the comparability, and even order on screen, can systematically affect choice. Previous researchers have also argued in favor of this point \parencite{trueblood2022attentional,hasan2025registered,evansImpactPresentationOrder2021}. 

There are numerous directions that this work could take beyond this dissertation. I could generalize the experimental and modeling paradigm of Experiment 2 to other context effects (e.g., compromise, similarity). I am also interested in measuring correlations between option valuations at the individual participant level, which is currently limited by the quantity of data available. 

Regarding best-worst choice, future work should explore models of best-worst choice that can be used when the independence assumption is violated. Exploration, or development is necessary, is beyond the scope of this dissertation. However, given the numerous applied uses for best-worst choice, this avenue of research would be important and fruitful.

In the future, I plan on continuing the line of research begun in Experiment 4. For example, I have already conducted a binary-ternary version of the choice phase from Experiment 4. I also plan on rnuning an experiment collecting both prices and choices in binary and ternary choice sets. This work will significantly add to the preferential choice and context effects literature.

In addition to the experimental modifications discussed earlier, future work in comparability should generalize the paradigm to various choice types. Additionally, the effect observed in Experiment 5 was quite small; practically speaking, this have limited impact on actual choices. Future work should address the limitations of comparability in affecting choice.

\section{Conclusions}
In this dissertation, I have identified various forms of context dependence, in both perceptual and preferential choice. I provided theoretical explanations for these results in the form of a mathematical model of choice. To ensure falsifiability, I tested the model's predictions on out-of-sample data, to considerable success. This work will further the study of context effects and decision-making in general. 
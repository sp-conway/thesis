\section{General Discussion}

Decisions, even simple ones, may depend on context. Context effects occur when the relative choice for a particular option varies with situational properties. In particular, the set of other available affects choice. The attraction effect occurs when a decoy option boosts the choice share of a similar, superior target option. The repulsion effect, a reversal of the attraction effect, occurs when participants select the competitor more than the target. 

This dissertation has explored context dependence in choice, largely by studying the attraction and repulsion effect (Chapters 2-4) but also through studying context dependence induced by the presentation format of options in simple perceptual choice (Chapter 5).

In Chapter 2, a Thurstonian choice model was developed to measure the correlation between valuations in the attraction and repulsion effect. Experiment 1 demonstrated systematic discriminability issues in a 2AFC task in accordance with the Thurstonian model. In Experiment 2 participants provided both valuations (area judgments for rectanges) and choices (participants selected the largest rectangle from each ternary choice set). The model, conditional on parameters estimated from the judgment data, was used to make predictions for the choice data. When the $\rho_{TD}$ parameter is greater than both the $\rho_{TC}$ and $\rho_{CD}$ parameters, the Thurstonian model can parsimoniously account for the repulsion effect without invoking higher-level decision processes (Experiment 2). Conditional on the parameters the model cannot, however, account for the attraction effect, suggesting that higher-level decision processes may be required to explain this effect. 

In Chapter 3, the Thurstonian model was generalized to best-worst choice. In best-worst choice, participants select their most and least preferred options from a given choice set. Given the stimuli from Experiment 2 (a set of target, competitor, and decoy rectangles in a perceptual choice experiment), the model predicts a non-monotonic relationship between best and worst choice probabilities. Specifically, the model predicts that the target is both less likely to be chosen as worst, compared to the competitor, and also less likely to be chosen as best. This effect is not predicted by the maxdiff model, a highly influential of best-worst choice, which typically assumes independently distributed utilities. These results were verified with empirical data in Experiment 3.

Chapter 4 generalized the paradigm and model from Chapter 2 to preferential choice. In Experiment 4, on each experimental trial, participants saw three consumer products (e.g., microwave ovens, laptops) and assigned each one a viable selling price. In later experimental trials, they saw the same three options and selected the option they preferred the most. Data showed that the similarity, and comparability, of the target and decoy options, appears to generalize across choice types and can be reliably measured using Pearson correlations. The choice data also replicated previous researchers' choice results \parencite{banerjeeFactorsThatPromote2024}, albeit with limitations described in the text.

The model from Experiment 4 is able to qualitatively account for the repulsion effect, there is one crucial limitation here; the model accounts for the effect through the correlation between target and decoy evaluations, which causes the decoy to take choice shares away from the target. In preferential choice, and unlike perceptual choice, participants seldom if ever select the decoy. In the absence of these correlations, or if all correlations are equal (i.e., $\rho_{TD}=\rho_{TC}=\rho_{CD}$), the model will be unable to predict the effect\footnote{See the diagonal line from Figure~\ref{fig:3d_model} for evidence of this result.}. Thus, this form of the repulsion effect may be due to higher-level decision processes, or a hitherto unidentified low-level process. Nonetheless, the demonstration of target-decoy correlations, and indeed target-competitor correlations, is a novel result. 

Though other researchers have proposed correlations as a measure of the similarity between options in a choice set \parencite{kamakura1984predicting,natenzon2019random}, these studies was the first (to my knowledge) to systematically measure these correlations using valuations, incorporate them into a Thurstonian choice model, and connect this model to choices obtained from the same experimental participants.

Chapter 5 demonstrated a different form of context dependence, where choice systematically varies based on option comparability. In Chapter 5, Given a \textit{symmetrically dominated decoy} option, placing a focal target option in a nearby position, such that participants can more easily compare it to the decoy, \textit{decreased} the target's choice share. 

The results of this dissertation also have important methodological considerations. In particular researchers should carefully consider the assumptions made when designing and analyzing experiments. For example, the experiments of \textcite{spektorWhenGoodLooks2018b} contain a crucial assumption: that because participants chose the target more often the decoy, the repulsion effect observed in these experiments is a qualitative reversal of the attraction effect rather than just an empirical one. Chapter 3 showed that the independence assumption of the maxdiff choice model is incorrect (at least in some cases) and a failure to consider whether the stimuli of a given experiment can cause these violations may lead to incorrect conclusions about participants' preferences. The results of Chapter 5 show that the comparability, and even order on screen, can systematically affect choice. Previous researchers have also argued in favor of this point \parencite{trueblood2022attentional,hasan2025registered,evansImpactPresentationOrder2021}. 

There are numerous directions that this work could take beyond this dissertation. Future work could generalize the experimental and modeling paradigm of Experiment 2 to other context effects (e.g., compromise, similarity). Researchers should also correlations between option valuations at the individual participant level, which is currently limited by the quantity of data available. 

Regarding best-worst choice, future work should explore models of best-worst choice that can be used when the independence assumption is violated. Exploration, or development is necessary, is beyond the scope of this dissertation. However, given the numerous applied uses for best-worst choice, this avenue of research may improve researchers' ability to identify participants' preferences. 

Future work should also continue the line of research begun in Experiment 4. For example, research could collect both choice and pricing data from various binary and ternary sets. 

In addition to the experimental modifications discussed earlier, future work in comparability should generalize the paradigm to various choice types. Additionally, the effect observed in Experiment 5 was quite small; practically speaking, this may have limited impact on actual choices. Future work should address the limitations of comparability in affecting choice.

\section{Conclusions}
This dissertation has identified various forms of context dependence, in both perceptual and preferential choice. The research has provided theoretical explanations for these results in the form of a mathematical model of choice. To ensure falsifiability, The model's predictions were tested on out-of-sample data, to varying degrees of success. This work will further the study of context effects and decision-making in general. 
\section{Introduction}
Thus far in the dissertation, I have focused on perceptual choice. This allowed me to reconcile conflicting findings from other researchers \parencite{spektorWhenGoodLooks2018b,trueblood2013not}. It also allowed me to develop a model of choice from the ground up, in a simplified choice environment.

However, many decision theorists, in particular those who study context effects, are interested in a wide variety of decision types. For example, the original demonstration of the attraction effect came from the marketing literature \parencite{huberAddingAsymmetricallyDominated1982d}. In this chapter, I generalize the paradigm from Experiment 2 to a consumer choice scenario. 

In Experiment 2, I collected psychophysical ratings and used those to estimate the parameters of a choice model, which I then applied to make predictions for choices in the same experiment. To test this approach in consumer choice, it is necessary to collect continuous ratings from participants in response to consumer stimuli. In Experiment 4, I collect both pricing data (the best continuous measure for consumer stimuli) and choices.

In most (but not all) studies of consumer choice, researchers collect choice data rather than ratings. There are good reasons for this. The literature on willingness to pay (WTP; the largest amount a given consumer would be willing to pay for a particular product) has shown that, when responding to hypothetical survey questions, participants tend to over-estimate their WTP by a sizeable aomunt \parencite{breidertREVIEWMETHODSMEASURING2006,schmidtAccuratelyMeasuringWillingness2020}, \parencite[c.f.~]{miller2011should}. 

These concerns, while crucial to applied researchers, are not relevant to the current study, as we are interested in participants' relative rather than absolute pricing. As in Experiment 2, where we were concerned with whether participants' estimates of perceived size increased with absolute size (regardless of how it deviated from actual size), we are interested in a pricing metric that increases in value. 

Other researchers have demonstrated context effects with ratings measures. \textcite{wedellUsingJudgmentsUnderstand} collected Likert scale attractiveness ratings for attraction effect stimuli, generally finding that the presence of a decoy increased mean ratings for a target option. \textcite{windschitl2004dud} asked participants to judge the likelihood of various events (also on a Likert scale). They found that the presence of a "dud" (highly unlikely) alternative increased participants' ratings of focal options. \textcite{caiWhenAlternativeHypotheses2023} demonstrated similar effects by collecting continuous probability judgments.

To my knowledge, however, there has been no research systematically connecting valuations and choices in a single experiment through application of a choice model. Thus, I seek to collecting continuous (pricing) ratings to estimate the parameters $\boldsymbol{\mu}$ and $\boldsymbol{\Sigma}$ for the choice model from Chapter 2 and use it to predict consumer choice data from the same experiment.

\subsection{Previous Research}
Previously, I showed that the model could capture the repulsion effect in perceptual choice \parencite{spektorWhenGoodLooks2018b}. I am now interested in whether the model can capture the repulsion effect in preferential choice. However, research on the repulsion effect is relatively sparse and atheoretical \parencite{liaoInfluenceDistanceDecoy2021,simonson2014vices,spektorRepulsionEffectPreferential2022}.  